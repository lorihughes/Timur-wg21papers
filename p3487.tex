\input{wg21common}

% Footnotes at bottom of page:
 \usepackage[bottom]{footmisc} 

% Table going across a page: 
 \usepackage{longtable}

 % Start sections at 0
% \setcounter{section}{-1}

% color boxes
\usepackage{tikz,lipsum,lmodern}
\usepackage[most]{tcolorbox}

%%%%%%%%%%%%%%%%%%%%%%%%%%%%%%%%%%%%%%%%%%%%%%%%

%TABLE OF CONTENTS SETTINGS

\usepackage{titlesec}
\usepackage{tocloft}

% Custom ToC layout because the default sucks
\cftsetindents{section}{0in}{0.24in}
\cftsetindents{subsection}{0.24in}{0.34in}
\cftsetindents{subsubsection}{0.58in}{0.44in}

% Needed later to reduce the ToC depth mid document
\newcommand{\changelocaltocdepth}[1]{%
  \addtocontents{toc}{\protect\setcounter{tocdepth}{#1}}%
  \setcounter{tocdepth}{#1}%
}

\setcounter{tocdepth}{3}

%%%%%%%%%%%%%%%%%%%%%%%%%%%%%%%%%%%%%%%%%%%%%%%%

%POLLS

\definecolor{pollFrame}{rgb}{0,.718,0}
\definecolor{pollBG}{rgb}{.5,1,.5}

\newtcolorbox{wgpoll}[1]{colframe=pollFrame,colback=pollBG!20!white,title={#1}}

\newcommand{\wgpollresult}[5]{%

  \vspace{2mm}
  \begin{tabular}{c | c | c | c | c} %
  SF  & F  & N  & A  & SA \\ %
  \hline %
  #1 & #2 & #3 & #4 & #5 \\ %

  \end{tabular}
  \vspace{2mm}  \\ %
}

%%%%%%%%%%%%%%%%%%%%%%%%%%%%%%%%%%%%%%%%%%%%%%%%

\begin{document}
\title{Postconditions odr-using a parameter \\ that may be passed in registers}
\author{
Timur Doumler \small(\href{mailto:papers@timur.audio}{papers@timur.audio}) \\
Joshua Berne \small(\href{mailto:jberne4@bloomberg.net}{jberne4@bloomberg.net}) \\
}
\date{}
\maketitle

\begin{tabular}{ll}
Document \#: & D3487R0 \\
Date: &2024-11-04 \\
Project: & Programming Language C++ \\
Audience: & SG21, EWG
\end{tabular}

\begin{abstract}
This paper considers the case where a non-reference parameter is odr-used in the predicate of a precondition or postcondition assertion and is eligible to be passed via registers. To enable caller-side checking of preconditions and postconditions, we need to add a provision to the the Contracts MVP \cite{P2900R10} that allows the check to observe either the caller-side or the callee-side version of the parameter object. However, for postconditions, this can lead to surprising behaviour. We propose several alternatives for how to address this problem.
\end{abstract}

%%%%%%%%%%%%%%%%%%%%%%%%%%%%%%%%%%%%%%%%%%%%%

%\tableofcontents*
%\pagebreak

%\section*{Revision history}

%Revision 0 (2024-04-16)
%\begin{itemize}
%\item Original version
%\end{itemize}

%\pagebreak

%%%%%%%%%%%%%%%%%%%%%%%%%%%%%%%%%%%%%%%%%%%%%

This paper is the second part of a trilogy of papers dealing with known issues in the Contracts MVP \cite{P2900R10} regarding postconditions odr-using non-reference function parameters:
\begin{itemize}
\item \cite{D3484R1} Postconditions odr-using a parameter modified in an overriding function;
\item \cite{D3487R0} Postconditions odr-using a parameter that may be passed in registers;
\item \cite{D3489R0} Postconditions odr-using a parameter of dependent type.
\end{itemize}
These issues should be considered together, and ideally resolved in a consistent way.

\section{Background}
\label{bg}

For efficiency reasons, the major ABIs used for implementations of C++ allow objects to be passed to a function and returned from a function via \emph{registers} when the type of the object satisfies certain requirements. The C++ Standard accommodates such passing and returning via registers in \href{https://timsong-cpp.github.io/cppwp/n4950/class.temporary#3}{[class.temporary]/3} as follows:

\begin{adjustwidth}{0.5cm}{0.5cm}
When an object of class type \tcode{X} is passed to or returned from a function, if \tcode{X} has at least one eligible copy or move constructor ([special]), each such constructor is trivial, and the destructor of \tcode{X} is either trivial or deleted, implementations are permitted to create a temporary object to hold the function parameter or result object. The temporary object is constructed from the function argument or return value, respectively, and the function's parameter or return object is initialised as if by using the eligible trivial constructor to copy the temporary (even if that constructor is inaccessible or would not be selected by overload resolution to perform a copy or move of the object).
\end{adjustwidth}

Effectively, such objects passed and returned via registers do not exist in memory and do not have an address; their value is instead accessed by materialising a temporary copy. In C++ today, this specification peculiarity causes no friction, because there is (with one minor exception\footnote{An object that meets the requirements to be passed in a register may still have a user-provide constructor that may observe its address through the use of \tcode{this}, and that address may then differ from the address seen for the parameter within the function body.}) no context where the pre-temporary copy versions of these objects could be directly observable by the user. 
The current wording does not say it explicitly, but there is an assumption within it that, once a temporary has been created to ``hold the function parameter or result object'', that temporary will be referred to whenever the name that denotes the object is used going forward.

In practice in C++ today, that is always the case. However, the Contracts MVP \cite{P2900R10} adds function-contract assertions (precondition and postcondition assertions). Depending on how we specify them, they could create a new context where the pre-temporary copy versions of  parameter objects and/or return objects are not only observable, but usable by name. We therefore must clarify exactly what semantics we want to allow in those cases, and what that means for implementations and for users.

\section{Discussion}

\subsection{Caller-side checking}

One of the design goals of \cite{P2900R10} is to allow the implementation to perform precondition and postcondition checks either callee-side or caller-side. A discussion of implementation strategies can be found in \cite{P3267R0} and \cite{P3321R0}; a discussion of the motivation and use cases for both caller-side and callee-side checks can be found in \cite{P3228R1}, \cite{P3264R1}, \cite{P3270R0}, \cite{P3321R0}, and references therein. We provide a very brief summary below,.

For callee-side checks, the compiler would emit code to perform the precondition and postcondition checks when compiling the definition of the function. For caller-side checks, the compiler would instead emit code around the function call to perform the checks. Note that the precondition and postcondition assertions are part of the function declaration and thus known at every call site.

Callee-side checks can be emitted in all cases. On the other hand, caller-side checks cannot be emitted in certain cases. One such case is an indirect call (e.g., through a pointer to function or pointer to member function), since the compiler does not know at the call site which concrete function will be called. Another such case is an ABI that requires function parameters to be destroyed callee-side (e.g., the Microsoft ABI), which means that postconditions cannot be checked caller-side without an ABI break, as postconditions must be evaluated before destruction of function parameters.

Even though not all checks can be performed caller-side in all cases, the ability to perform at least \emph{some} caller-side checks is important. With this ability, the user can enable contract checks to diagnose defects when working with a library that has been compiled with contract checks off (it can often be too costly or outright impossible to recompile the library with contract checks on and re-link the program). Caller-side checks also enable an implementation of contract checks on virtual functions as specified in \cite{P2900R10}: the caller-facing contract (that of the statically called function) can be checked caller-side, while the callee-facing contract (that of the final overrider selected by virtual dispatch) can be checked callee-side. Note that only the caller knows the caller-facing contract in this case.\footnote{How one could implement checking the caller-facing contract of a virtual function call on an ABI that requires function parameters to be destroyed callee-side, without forcing an ABI break, is currently still an open question, but this problem is only tangentially related to the problem discussed in this paper.}

\subsection{Preconditions and parameters}
\label{pre}

Precondition checks can only observe objects passed to a function, i.e., the function parameters, not objects returned from the function. \cite{P2900R10} currently does not contain an explicit provision that would allow precondition checks to observe the pre-temporary copy parameter objects. It follows therefore from [class.temporary] that precondition checks must observe the same parameter object as the function body. This makes caller-side precondition checks unimplementable with the current specification.

We can fix this problem by adding an explicit provision to \cite{P2900R10} that a precondition check is implicitly allowed to observe either the pre-temporary parameter object or the temporary copy. If the precondition check observes the pre-temporary object, it will do so before the copy is made to pass the object into a function. If it observes the temporary copy, it will observe the same object as the function body. In either case, there is no problem.

\subsection{Postconditions and the return object}

Unlike precondition checks, postcondition checks can observe both parameters and the return object. \cite{P2900R10} contains an explicit provision that allows postcondition checks to observe either the caller-side or the callee-side version of the return object:

\begin{adjustwidth}{0.5cm}{0.5cm}
If the implementation is permitted to introduce a temporary object for the return value
([class.temporary]), the result name may instead denote that temporary object.
\end{adjustwidth}

This provision is directly observable. Consider:

\begin{codeblock}
class X { /* ... */ };

X f(const X* ptr) post(r: &r == ptr) {
  return X{};
}

int main() {
  X x = f(&x);
}
\end{codeblock}

In the example above, if \tcode{X} is \emph{not} a type eligible to be passed via registers, the postcondition check of \tcode{f} is guaranteed to pass, because \tcode{r} must denote the return object \tcode{x} in \tcode{main()}; however, if \tcode{X} \emph{is} a type eligible to be passed via registers, the postcondition check may\footnote{In practice, whether this check fails will depend on both optimization levels and whether \tcode{f} is inlined into \tcode{main()}.} fail, because \tcode{r} may instead denote a temporary object.

This behaviour may be surprising to a user not familiar with the rules for returning objects via registers, but there is no actual problem --- this behaviour is an exact mirror image of parameters in postcondition assertions. Postcondition checks that refer to the return object may therefore be implemented either caller-side or callee-side with the current specification.

\subsection{Postconditions and parameters}

The third and last case to consider is a postcondition assertion odr-using a non-reference parameter:

\begin{codeblock}
X* ptr;

void f(const X x) post (ptr == &x) {
  ptr = &x;
}
\end{codeblock}

If \tcode{X} is a type eligible to be passed via registers, is this postcondition guaranteed to pass or not?

Just like for precondition assertions, \cite{P2900R10} currently does not contain an explicit provision that would allow postcondition checks to observe the pre-temporary copy parameter objects. It follows therefore from [class.temporary] that postcondition checks, like precondition checks, must observe the same parameter object as the function body; the postcondition assertion above is guaranteed to pass.

This makes caller-side precondition checks unimplementable with the current specification (other reasons why they might be unimplementable, in particular an ABI that requires function parameters to be destroyed callee-side, notwithstanding).

Just like for the previous two cases, we could add an explicit provision to \cite{P2900R10} that a postcondition check is implicitly allowed to observe either the pre-temporary parameter object or the temporary copy. However, unlike for preconditions, for postconditions we now run into the problem that the temporary copy is made when the function is called, but the postcondition assertion is evaluated when the function returns, and arbitrary code can be executed in between. 

\cite{P2900R10} requires every parameter odr-used in a postcondition assertion  to be declared \tcode{const} on all declarations of the function, and requires that function to not be a coroutine, which guarantees that the parameter object that the function body observes is not modified between the function call and its return. However, if the parameter is passed in registers, and the postcondition observes the pre-temporary copy parameter object, these two versions of the parameter object could diverge. This has surprising consequences and renders the postcondition's actual meaning significantly more difficult to reason about. 

In particular, even if the parameter object is \tcode{const}, the function body could still modify any \tcode{mutable} members of that class. If the postcondition assertion is allowed to observe the pre-temporary copy version of the object, it will not see such modifications.

Now, of course we should not write postconditions that directly depend on such mutable state anyway, and if we do, we have already shot ourselves in the foot. But the problem at hand is more subtle: we may have a type whose correctness is connected to that mutable state.

Consider a class \tcode{RandomInteger} holding a random integer value:

\begin{codeblock}
class RandomInteger {
  int _value = rand();
public:
  int value() const { 
    return _value; 
  }
};
\end{codeblock}

The value is computed once when an object of type \tcode{RandomInteger} is created and does not change afterwards. This value is accessible via a public \tcode{value()} member function, which is marked \tcode{const} because it does not change the \emph{observable} state of the object --- it always returns the same value throughout its lifetime.

As an implementation detail, we might compute the value lazily when \tcode{value()} is called for the first time, and cache it afterwards, with no observable change in behaviour:
 \begin{codeblock}
class RandomInteger {
  mutable bool _computed = false;
  mutable int  _value;
public:
  int value() const {
    if (!_computed) {
      _value = rand();
      _computed = true;
    }
    return _value;
  }
};
 \end{codeblock}
Note that our \tcode{RandomInteger} class, as defined above, has a trivial copy constructor and a trivial destructor and is therefore eligible to be passed via registers. This leads to a new footgun: 
 \begin{codeblock}
int f(const RandomInteger i)
post(r: r & i.value() == 0) {
  return ~i.value();
}
 \end{codeblock}
If there is no guarantee that the \tcode{i} in the postcondition assertion refers to the same object as the \tcode{i} in the function body, this code will break. The postcondition assertion will see a different integer value returned from \tcode{i.value()} than the body of the function, and thus fail where it should pass or vice versa, even though for any user reading this code without a deep understanding of objects being passed in registers will see nothing obviously incorrect with this code.

Such code works in C++ today because after the parameter object has been packed into registers and passed to the function,  the original object will never be touched by anything again (remember that the type also needs to have a trivial or deleted destructor, not just an eligible trivial copy or move constructor). However, allowing a postcondition check to observe the original object --- which is required to enable caller-side postcondition checks --- changes that, which creates the footgun above.

We are aware of seven possible options for dealing with this problem. These options are, from most to least restrictive:

% custom enumerators with R prefix:
\renewcommand\labelenumi{R\arabic{enumi}.}
\renewcommand\theenumi\labelenumi

\begin{enumerate}
\item Remove postcondition assertions from \cite{P2900R10} entirely.
\item Remove the ability to odr-use \emph{any} function parameters in postcondition predicates.
\item Remove the ability to odr-use any \emph{non-reference} function parameters in postcondition predicates.
\item Remove the ability to odr-use non-reference function parameters in postcondition predicates unless the parameter is of scalar type.\footnote{Scalar types are arithmetic types, enumeration types, pointer types, pointer-to-member types, \tcode{std::nullptr_t}, and $cv$-qualified versions of these types. Arithmetic types are integral and floating-point types; integral types include character types and \tcode{bool}.}
\item Remove the ability to odr-use non-reference function parameters in postcondition predicates if the type of the parameter satisfies the requirements for being passed via registers.
\item Clarify that when a non-reference function parameter is odr-used in a postcondition predicate, the corresponding \grammarterm{id-expression} must refer to the same object as it does in the function body (status quo in \cite{P2900R10}).
\item Add an explicit provision that, when a non-reference function parameter is odr-used in a postcondition predicate and the type of the parameter satisfies the requirements for being passed in registers, the corresponding \grammarterm{id-expression} may refer either to the same object as it does in the function body or to the return object at the call site.
\end{enumerate}

We enumerated the options with an ``R'' prefix (for ``registers''), to distinguish them from the options from \cite{D3484R1} that have a ``V'' prefix (for ``virtual'') and the options from \cite{D3489R0} that have a ``D'' prefix (for ``dependent'').

Below we discuss the tradeoffs of each option.

\subsection*{Option R1}

Option~R1 would be a rather drastic measure at this point. However, consider that postcondition assertions are significantly more difficult to specify than preconditions (see \cite{P1323R2}, \cite{P3007R0}, and \cite{P3098R0}), and unlike preconditions, postcondition assertions have so far already generated several known issues that needed fixing in the specification of \cite{P2900R10} (see \cite{P3387R0}, \cite{P3460R0}, \cite{P3483R0}, \cite{D3484R1}, and \cite{D3489R0}). Option~R1 would remove \emph{all} known and unknown footguns from postcondition assertions by removing the feature itself.

We believe that \cite{P2900R10} would still be viable and useful without postcondition assertions. Postcondition assertions have significantly fewer uses than precondition assertions, and their value can to a certain extent also be achieved with good unit test coverage. 

For the record, the option of removing postcondition assertions from the Contracts MVP was once before polled in SG21:

\begin{wgpoll}{SG21 Poll, Teleconference 2021-12-14}
Postconditions should be in the MVP at this time.
\wgpollresult{1}{7}{3}{4}{1}
Result: Marginal consensus (if at all).
\end{wgpoll}

\subsection*{Option R2}

Option~R2 is likewise less than ideal in our opinion, because it significantly limits the set of postconditions we can write, and thus significantly limits the usefulness of the feature, until we can add postconditions captures \cite{P3098R0} to the Standard.

This option becomes more appealing if we include \cite{P3098R0} in the first version of Contracts that we ship. Postcondition captures would never be referencing parameters in a place they cannot be referenced today, so they would not be impacted by this issue at all. However, capturing parameters for use in a postcondition predicate incurs the cost of a copy, which in many cases is not conceptually necessary, thus violating the ``do not pay for what you do not use'' design principle of C++ (see also \cite{D3484R1} Option~R1, which suffers from the same issue).


\subsection*{Option R3}

Option~R3 is similar to Option~R2, except that it allows reference parameters, which are not affected by any of the issues surrounding copies of objects and are not affected by postcondition captures as proposed in \cite{P3098R0}.

However, allowing only reference parameters still significantly limits the set of postconditions we can write. In addition, it encourages users to pass parameters by-reference instead by-value as this would be the only way to make the postcondition assertion compile, which can lead to more error-prone and less efficient code than the normally recommended pass-by-value. We therefore do not consider Option~R3 to be an improvement over Option~R2.

\subsection*{Option R4}

Option~R4 would allow by-value parameters of types that are not affected by the footgun and cannot be changed such that they would be affected by the footgun, i.e., scalar types. This would already enable many more useful postconditions compared to Options R1 --- R3.

However, if we were to change the type of a parameter from a built-in type to a user-defined type, for example from \tcode{int} to \tcode{BigInt}, or from \tcode{double} to \tcode{std::complex<double>}, the postcondition would stop compiling, with no workaround available. Additionally,  Option~R4 would make it significantly harder to add postconditions to generic code, including any templates designed to work with both built-in and user-defined types (which includes the entire STL and vast amounts of other generic libraries).

\subsection*{Option R5}

Option~R5 is less restrictive still: it limits postconditions to odr-using parameters of types that are not affected by the footgun. However, most users will not be familiar with the rules around types eligible to be passed in registers, which means that the compiler error they get will be very confusing to most users. Worse, the definition of user-defined types can change over time, which makes this option brittle. That a particular type is \emph{not} eligible to be passed via registers is not something that code should guarantee to its clients indefinitely in all cases; conversely, making a type trivially copyable and/or movable and trivially destructible should not break client code.

We made a similar choice to not discriminate on particular type traits in the postcondition assertions of a function when we decided that whether a type is trivially movable should not affect whether a non-reference parameter of that type can be odr-used in the postcondition assertion of a coroutine (i.e., when we rejected \cite{P3387R0} Option 5b). Choosing Option~R5 here would be inconsistent with that design choice.

\subsection*{Option R6}

Option~R6 is the status quo. It is the only solution that both avoids the above footgun and is not a breaking change to \cite{P2900R10}: postconditions odr-using a \tcode{const} non-reference parameter of a non-coroutine function remain well-formed. Option~R6 is therefore arguably the optimal choice with respect to the immediate user experience when dealing with code such as the above.

However, Option~R6 also comes with a high cost: implementing caller-side postcondition checks remains impossible without an ABI break. The necessary ABI break to enable caller-side postcondition checks --- and by extension, to implement full contract checks on virtual functions as specified by \cite{P2900R10} --- would consist of no longer passing function parameters via registers if they are odr-used in the postcondition assertion. However, imposing an ABI break on all users who wish to add postcondition assertions to their functions would arguably do significant harm to the adoption of Contracts in the C++ ecosystem, which is why one of the fundamental design principles of the Contracts MVP (\cite{P2900R10} Principle 16) is to avoid such an ABI break.

\subsection*{Option R7}

Option~R7 makes the behaviour of postconditions with respect to objects passed to a function in registers consistent with the behaviour for objects returned from a function in registers, as well as with the behaviour of preconditions as proposed in Section~\ref{pre}. Option~R7 is therefore arguably the optimal choice with respect to having a straightforward specification and implementation of the feature, enabling caller-side checking, avoiding ABI issues, not making any user code unnecessarily ill-formed, and not imposing any unnecessary run-time cost on the user.

However, the tradeoff of Option~R7 is that it adds the above footgun to the C++ language. Note that the footgun only occurs in rare edge cases, in particular when a \tcode{const} object eligible to be passed via registers is used as a non-reference parameter and its type relies on mutable state for its correctness, and a postcondition assertion would break if it happens to observe an earlier version of that mutable state. Note further that such cases could potentially be mitigated by an implementation issuing a warning if a type eligible to be passed in registers has \tcode{mutable} members (including in any of its subtypes), is used as the type of a non-reference function parameter, and that parameter is odr-used in a postcondition assertion of that function or another function that that function overrides.

\section{Proposal}

With regards to parameters odr-used in \emph{preconditions}, we propose to add a provision to \cite{P2900R10} that a precondition check is implicitly allowed to observe either the pre-temporary parameter object or the temporary copy, as discussed in Section~\ref{pre}.

With regards to parameters odr-used in \emph{postconditions}, we believe that Options R1 --- R7 are all worth considering and the tradeoffs of each option are sufficiently clear. We therefore propose all of these options to determine which option has more consensus in SG21. 

Note that choosing Option~R1 would leave the door open to adopting any of the other options without breaking changes at some point in the future; Option~R2 could be evolved towards Options R3 --- R7, but not R1, etc, until Option~R5, which could only be evolved towards Options R6 and R7 but not any others. Finally, Options R6 and R7 are mutually exclusive and cannot be evolved towards any of the other options without breaking changes once adopted.

Note further that choosing Options R1 --- R3 would also remove the source of the issues discussed in the two companion papers \cite{D3484R1} and \cite{D3489R0}. 

Note finally that choosing Option R4 would be consistent with a possible relaxation of the rule for postconditions on coroutines to also accept non-reference parameters of scalar type.

\section{Wording}

The proposed wording changes are relative to the wording proposed in \cite{P2900R10}.

\subsection*{Preconditions}

Modify [dcl.contract.func] as follows:

\begin{adjustwidth}{0.5cm}{0.5cm}
\added{If the predicate of a precondition assertion of a function odr-uses ([basic.def.odr]) a non\-reference parameter of that function, and the implementation is permitted to introduce a temporary object for the parameter object value ([class.temporary]), it is unspecified whether the corresponding \grammarterm{id-expression} in the predicate of the precondition assertion denotes that temporary object or the original parameter object.} If the predicate of a postcondition assertion of a function odr-uses \removed{([basic.def.odr]) }a non\-reference parameter of that function, that parameter shall be declared \tcode{const} and shall not have array or function type. 
\begin{note}
This requirement applies even to declarations
that do not specify the \grammarterm{postcondition-specifier}. Arrays and functions are still usable when declared with the equivalent pointer types ([dcl.fct]).
\end{note}
\begin{example}
\tcode{[...]}
\end{example}
\end{adjustwidth}

\subsection*{Postconditions --- Option R1}

Remove all wording that relates to:
\begin{itemize}
\item The \tcode{post} identifier with special meaning;
\item The \emph{postcondition-specifier} and \emph{result-name-introducer} grammar non-terminals;
\item Postcondition assertions and result names;
\item The \tcode{post} enumeration value in \tcode{std::contracts::assertion_kind}.
\end{itemize}
The exact wording diff is not provided here due to its length.

\subsection*{Postconditions --- Option R2}

Modify [dcl.contract.func] as follows:

\begin{adjustwidth}{0.5cm}{0.5cm}
If the predicate of a postcondition assertion of a function odr-uses ([basic.def.odr]) a
\removed{non\-reference }parameter of that function, \removed{that parameter shall be declared \tcode{const} and shall not have array or function type}\added{the program is ill-formed}.
\begin{note}
This requirement applies even to declarations
that do not specify the \grammarterm{postcondition-specifier}. Arrays and functions are still usable when declared with the equivalent pointer types ([dcl.fct]).
\end{note}
\begin{example}
\tcode{[...]}
\end{example}
\end{adjustwidth}

\subsection*{Postconditions --- Option R3}

Modify [dcl.contract.func] as follows:

\begin{adjustwidth}{0.5cm}{0.5cm}
If the predicate of a postcondition assertion of a function odr-uses ([basic.def.odr]) a
non-reference parameter of that function, \removed{that parameter shall be declared \tcode{const} and shall not have array or function type}\added{the program is ill-formed}.
\begin{note}
This requirement applies even to declarations
that do not specify the \grammarterm{postcondition-specifier}. Arrays and functions are still usable when declared with the equivalent pointer types ([dcl.fct]).
\end{note}
\begin{example}
\tcode{[...]}
\end{example}
\end{adjustwidth}

\subsection*{Postconditions --- Option R4}

\begin{adjustwidth}{0.5cm}{0.5cm}
If the predicate of a postcondition assertion of a function odr-uses ([basic.def.odr]) a
non-reference parameter of that function, that parameter shall be declared \tcode{const} and shall \removed{not have array or function type}\added{have scalar type ([basic.types.general])}.
\begin{note}
This requirement applies even to declarations
that do not specify the \grammarterm{postcondition-specifier}.\removed{ Arrays and functions are still usable when declared with the equivalent pointer types ([dcl.fct]).}
\end{note}
\begin{example}
\tcode{[...]}
\end{example}
\end{adjustwidth}

\subsection*{Postconditions --- Option R5}

Modify [dcl.contract.func] as follows:

\begin{adjustwidth}{0.5cm}{0.5cm}
If the predicate of a postcondition assertion of a function odr-uses ([basic.def.odr]) a
non-reference parameter of that function, that parameter shall be declared \tcode{const}\added{, }\removed{and }shall not have array or function type\added{, and shall not have a type for which the implementation is permitted to create a temporary object to hold the function parameter ([class.temporary])}.
\begin{note}
This requirement applies even to declarations
that do not specify the \grammarterm{postcondition-specifier}. Arrays and functions are still usable when declared with the equivalent pointer types ([dcl.fct]).
\end{note}
\begin{example}
\tcode{[...]}
\end{example}
\end{adjustwidth}

\subsection*{Postconditions --- Option R6}

Modify [dcl.contract.func] as follows:

\begin{adjustwidth}{0.5cm}{0.5cm}
If the predicate of a postcondition assertion of a function odr-uses ([basic.def.odr]) a
non-reference parameter of that function, that parameter shall be declared \tcode{const} and shall not have array or function type.
\begin{note}
This requirement applies even to declarations
that do not specify the \grammarterm{postcondition-specifier}. Arrays and functions are still usable when declared with the equivalent pointer types ([dcl.fct]).
\added{An \emph{id-expression} that denotes a non-reference parameter in the predicate of a postcondition assertion denotes the same object as in the function body, even if the implementation is permitted to introduce a temporary object for the parameter object value ([class.temporary]).}
\end{note}
\begin{example}
\tcode{[...]}
\end{example}
\end{adjustwidth}

\subsection*{Postconditions --- Option R7}

Modify [dcl.contract.func] as follows:

\begin{adjustwidth}{0.5cm}{0.5cm}
If the predicate of a postcondition assertion of a function odr-uses ([basic.def.odr]) a
non-reference parameter of that function, that parameter shall be declared \tcode{const} and shall not have array or function type.
\begin{note}
This requirement applies even to declarations
that do not specify the \grammarterm{postcondition-specifier}. Arrays and functions are still usable when declared with the equivalent pointer types ([dcl.fct]).
\end{note}
\begin{example}
\tcode{[...]}
\end{example}

\added{If the implementation is permitted to introduce a temporary object for the parameter object value ([class.temporary]), it is unspecified whether the corresponding \emph{id-expression} in the predicate of a postcondition assertion denotes that temporary object or the original parameter object. [ \emph{Note:} It follows that, for objects that can be passed in registers, the postcondition assertion might not see any modifications of \tcode{mutable} subobjects ([dcl.stc]) of the parameter object performed by the function or a function overriding it. --- \emph{end note} ]}
\end{adjustwidth}

%%%%%%%%%%%%%%%%%%%%%%%%%%%%%%%%%%%%%%%%%%%%%

\section*{Acknowledgements}
Thanks to Alisdair Meredith, John Lakos, Jens Maurer, and Mungo Gill for their helpful remarks during drafting of this paper. Thanks to Oliver Rosten for his review of the paper.

%%%%%%%%%%%%%%%%%%%%%%%%%%%%%%%%%%%%%%%%%%%%%

% Remove ToC entry for bibliography
\renewcommand{\addcontentsline}[3]{}% Make \addcontentsline a no-op to disable auto ToC entry

%\renewcommand{\bibname}{References}  % custom name for bibliography
\bibliographystyle{abstract}
\bibliography{ref}

%%%%%%%%%%%%%%%%%%%%%%%%%%%%%%%%%%%%%%%%%%%%%


\end{document}
