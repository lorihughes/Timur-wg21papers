\input{wg21common}

% Footnotes at bottom of page:
 \usepackage[bottom]{footmisc} 

% Table going across a page: 
 \usepackage{longtable}

 % Start sections at 0
% \setcounter{section}{-1}

% color boxes
\usepackage{tikz,lipsum,lmodern}
\usepackage[most]{tcolorbox}

%%%%%%%%%%%%%%%%%%%%%%%%%%%%%%%%%%%%%%%%%%%%%%%%

%TABLE OF CONTENTS SETTINGS

\usepackage{titlesec}
\usepackage{tocloft}

% Custom ToC layout because the default sucks
\cftsetindents{section}{0in}{0.24in}
\cftsetindents{subsection}{0.24in}{0.34in}
\cftsetindents{subsubsection}{0.58in}{0.44in}

% Needed later to reduce the ToC depth mid document
\newcommand{\changelocaltocdepth}[1]{%
  \addtocontents{toc}{\protect\setcounter{tocdepth}{#1}}%
  \setcounter{tocdepth}{#1}%
}

\setcounter{tocdepth}{3}

%%%%%%%%%%%%%%%%%%%%%%%%%%%%%%%%%%%%%%%%%%%%%%%%

%POLLS

\definecolor{pollFrame}{rgb}{0,.718,0}
\definecolor{pollBG}{rgb}{.5,1,.5}

\newtcolorbox{wgpoll}[1]{colframe=pollFrame,colback=pollBG!20!white,title={#1}}

\newcommand{\wgpollresult}[5]{%

  \vspace{2mm}
  \begin{tabular}{c | c | c | c | c} %
  SF  & F  & N  & A  & SA \\ %
  \hline %
  #1 & #2 & #3 & #4 & #5 \\ %

  \end{tabular}
  \vspace{2mm}  \\ %
}

%%%%%%%%%%%%%%%%%%%%%%%%%%%%%%%%%%%%%%%%%%%%%%%%

\begin{document}
\title{Undefined behaviour is a contract violation}
\author{ Timur Doumler \small(\href{mailto:papers@timur.audio}{papers@timur.audio})  \\
Ga\v sper A\v zman \small(\href{mailto:gasper.azman@gmail.com}{gasper.azman@gmail.com})   \\
Joshua Berne \small(\href{mailto:jberne4@bloomberg.net}{jberne4@bloomberg.net})  
}
\date{}
\maketitle

\begin{tabular}{ll}
Document \#: & D3100R2 \\
Date: &2025-04-16 \\
Project: & Programming Language C++ \\
Audience: & EWG
\end{tabular}

\begin{abstract}
In this paper, we propose a generic framework for effective runtime mitigation of undefined behaviour in C++. We introduce \emph{implicit contract assertions}, which follow the same semantics as C++26 contract assertions \cite{P2900R14}, but are added implicitly by the compiler, rather than explicitly by the user. We further propose a mechanism for adapting \emph{labels} \cite{P3400R1} to control the evaluations semantics of implicit contract assertions. When a \emph{checked} semantic is chosen by the user, undefined behaviour is replaced by an implicit runtime check, followed by a call to the contract-violation handler and/or program termination.

We propose to apply this strategy across the entire C++ language, taking a major step towards removing undefined behaviour from C++. This paper is intended as a direct contribution to the core language undefined behaviour white paper \cite{P3656R0} which EWG intends to draft and ship in the C++26 timeframe.
\end{abstract}

%%%%%%%%%%%%%%%%%%%%%%%%%%%%%%%%%%%%%%%%%%%%%

\tableofcontents*
\pagebreak

%\section*{Revision history}

%Revision 0 (2024-04-16)
%\begin{itemize}
%\item Original version
%\end{itemize}

%\pagebreak

%%%%%%%%%%%%%%%%%%%%%%%%%%%%%%%%%%%%%%%%%%%%%

\section{Introduction}
\label{intro}

\subsection{Motivation and context}

Eliminating or at least meaningfully reducing the amount of unbounded \emph{undefined behaviour} is arguably the most critical objective for the future evolution of the C++ language. The ISO C++ committee has been continuously working in that direction. Common sources of undefined behaviour that have been removed in recent iterations of the C++ Standard include implicitly created objects \cite{P0593R6} for C++20; temporaries in range-based for loops \cite{P2012R2} for C++23; as well as trivial infinite loops \cite{P2809R3} and certain cases\footnote{If the variable is default-initialised, of scalar type, and has automatic storage duration.} of reading uninitialised values \cite{P2795R5} for C++26.

While these improvements are certainly laudable, they are case-by-case fixes that do not meaningfully change the perception of C++ as a fundamentally ``unsafe'' language.\footnote{See \cite{NSA2022}, \cite{CR2023}, \cite{CISA2023}, \cite{ONCD2024}. For a good contextualisation of the current discourse, see \cite{Sutter2024}. For recommendations regarding usage of the word ``safe'' in this context, see \cite{P3500R1} and \cite{P3578R0}.} To truly move the needle, we need consensus on a \emph{holistic} and \emph{actionable} strategy for mitigating undefined behaviour in C++.

On the committee level, no such consensus strategy exists yet. However, at the February 2025 meeting in Hagenberg, we agreed on a \emph{ship vehicle} for such a strategy: the \emph{core language UB white paper} \cite{P3656R0}, which we hope to draft and ship within the C++26 timeframe. This paper is intended as a direct contribution to that white paper.

As a first step, the editors of that white paper are currently working on enumerating all instances of core language undefined behaviour in C++, building on earlier work in \cite{P1705R1}. In this paper, we take a slightly different approach: we \emph{categorise} instances of core undefined behaviour according to several criteria that are relevant for choosing the optimal mitigation strategy (see Section~\ref{categoriesofub}).

The most important such criterion is whether a particular instance of undefined behaviour should be mitigated at \emph{compile time} (by making the offending construct ill-formed, and offering alternative constructs where needed), or at \emph{run time} (by making the behaviour of \emph{executing} the offending construct no longer undefined). From an engineering perspective, the former approach seems preferable (``shift left''); however, in many cases it is not feasible because necessary information is not available until runtime or due to the need to be backwards-compatible with existing C++ code.

In this paper, we consider these two approaches to be complementary to each other. We leave the compile-time approach up to other ongoing efforts such as Profiles \cite{P3081R2} and borrow checking \cite{P3390R0}. Instead, we focus entirely on instances of undefined behaviour that --- for whatever reason --- \emph{cannot} be mitigated at compile time and need to be mitigated at run time. We propose a generic and universal framework for such runtime mitigation that can be applied across the entire C++ language.

Our proposed framework builds on top of Contracts, a paradigm for reasoning about incorrect programs and identifying program defects, including those that can lead to undefined behaviour. For C++26, we adopted \cite{P2900R14}, an initial subset of functionality based on this paradigm. This initial subset contains three kinds of \emph{contract assertions}: \tcode{pre}, \tcode{post}, and \tcode{contract_assert}. All three are \emph{explicit} contract assertions: they are specified with explicit syntax and need to be intentionally added by the author of the code.

For example, the author of a vector-like class can add a precondition assertion to its subscript operator to guard against out-of-bounds access:

\begin{codeblock}
T& operator[] (size_t index)
  pre (index < size());
\end{codeblock}

By choosing a checked \emph{evaluation semantic} for this contract assertion (\emph{observe}, \emph{enforce}, or \emph{quick-enforce}) the user can opt into defined behaviour (program termination and/or a call to a global contract-violation handler which can itself be user-defined) when their vector is accessed out-of-bounds. Importantly, the user can also opt out of the runtime check by choosing an unchecked evaluation semantic (\emph{ignore}) in cases where the use case requires it.

This approach is most appropriate when it is up to the human to determine the conditions under which the program is considered \emph{correct} (i.e., its plain-language contract). On the other hand, explicit contract assertions are less ideal to identify situations where a core language construct exhibits undefined behaviour when evaluated (out-ouf-bounds access into a plain array, arithmetic overflow, etc.) because the defect will not be found unless an appropriate contract assertion is explicitly added. In other words, explicit contract assertions do nothing for \emph{existing} C++ code.

However, for any core language construct, determining whether undefined behaviour occurs during program execution does not require an explicit contract assertion: the conditions under which it occurs are specified in the C++ Standard. Mitigations that turn undefined into well-defined behaviour can therefore be inserted in an automated fashion by compilers (e.g., GCC's \tcode{-ftrapv} and \tcode{-fwrapv} flags) and sanitisers (e.g., ASan, UBSan).

Such mitigations are widely used today, but they also have a number of drawbacks: they are not \emph{complete} (they do not cover all undefined behaviour in C++); they are not portable across different compilers and toolchains; they do not share a common paradigm or terminology; and they exist entirely outside of the C++ Standard. The goal of this paper is to address all of these drawbacks by introducing a unified framework to the C++ Standard that both generalises and extends existing approaches.

\subsection{The basic idea}

The basic building block of our proposed framework is the introduction of \emph{implicit contract assertions}, which work like explicit contract assertions, but are added implicitly by the implemenation, rather than explicitly by the user.

As an example, let us consider indexing into a plain array rather than a user-defined vector-like class. Let us further assume for the purpose of this example that the size \tcode{N} of this array is statically known:

\begin{codeblock}
int main() {
int a[10] = { /* ... */ };
std::size_t i; std::cin >> i;
return a[i];
}
\end{codeblock}

In C++ today, the behaviour of this program is undefined if the value of \tcode{i} is not smaller than 10. At the same time, whether undefined behaviour will actually occur during execution cannot be determined at compile time because the value of \tcode{i} is not known until run time.

Instead of saying that out-of-bounds access into a plain array is undefined behaviour, we now say that access into a plain array has an \emph{implicit precondition assertion} that the index is not out-of-bounds. In other words, the program behaves as-if the compiler had wrapped every raw array subscript operation for which it statically knows the array bound \tcode{N} into an inline function with a precondition assertion:

\begin{codeblock}
template <typename T, std::size_t N>
T& __index_into_array(T (&a)[N], std::size_t i)
pre (i < N) {
  return a[i];
}
\end{codeblock}

Other than being an implicit precondition assertion generated by the compiler, \tcode{pre (i < N)} behaves the same as an explicit precondition assertion. That is, the user has the same choice of four evaluation semantics (\emph{ignore}, \emph{observe}, \emph{enforce}, or \emph{quick-enforce}) to specify the desired behaviour depending on the tradeoffs that are most suitable for their application.

Furthermore, when an out-of-bounds access is detected and the semantic is \emph{observe} or \emph{enforce}, the same contract-violation handler as for explicit contract assertions is called. The \tcode{contract_violation} object passed into the handler provides a way to programmatically distinguish explicit and implicit contract assertions and the kind of error that has occurred (see Section~\ref{libraryapi}).

Going beyond the above example, the key observation is that \emph{every instance} of undefined behaviour in C++ can in principle be mitigated by an appropriately specified implicit contract assertion. The generic algorithm for this transformation is to change every occurrence of “Operation $X$ has undefined behaviour if $A$ is not \tcode{true}'' to ``Operation $X$ has an implicit precondition that $A$ is \tcode{true}'' throughout the entire core language section of the C++ Standard.

In practice, these implicit contract assertions vary widely in terms of how computationally expensive they are, how much additional instrumentation they require; a thorough analysis can be found in Section~\ref{categoriesofub} of this paper. Notably, it is not always possible to express the predicate $X$ as a C++ expression (``\tcode{ptr} points to a valid object'' is an important counterexample). Yet, it is always possible to determine the value of the predicate $X$ somehow, at least in theory. Which checks are actually going to be offered by any given implementation, and which of these will be enabled by default, can be determined on a case-by-case basis. Different options offered by implementations (e.g., a debug build, a release build, or a build with a sanitiser enabled) can be mapped to different choices of evaluation semantic for every kind of implicit contract assertion. %TODO: maybe add reference to relevant section

Note that because implicit contract assertions are contract assertions, they benefit from the same extensions to the Contracts framework as explicit contract assertions. This includes in-source control of the evaluation semantic with arbitrary granularity using \emph{labels} (see Section~\ref{labels}) and the introduction of additional evaluation semantics to cover more use cases (see Section~\ref{assume}).

\subsection{History and related work}

The first revision (R0) of this paper was published in May 2024. Following informal discussions at the St. Louis meeting, the paper was revised (R1) and presented to SG21 (Contracts study group) at the Wroc{\l}aw meeting. SG21 voted \emph{unanimously} in favour of our direction:
\vspace{1mm}
\begin{wgpoll}{{SG21 Poll 6, Wroc{\l}aw, 2024-11-22}}
We support the direction of P3100R1 and encourage the authors to come back with a fully specified proposal.
\wgpollresult{19}{6}{0}{0}{0}
Result: Consensus
\end{wgpoll}

The present revision (R2) takes into account the above poll, the adoption of \cite{P2900R14} into the C++26 working paper, the decision to publish the white paper \cite{P3656R0}, and other progress made at the Hagenberg meeting.

In parallel, runtime checks guarding against core language undefined behaviour is also being considered as part of the ongoing work on Profiles. The latest published revision of the Profiles at the time of writing is \cite{P3081R2}; it proposes runtime checks guarding against out-of-bounds array access, null pointer dereference, and arithmetic overflow. \cite{P3436R1} is a related paper outlining the proposed strategy. However, the counter-papers \cite{P3543R0} and \cite{P3558R1} argue that the insertion of implicit runtime checks does not belong into the Profiles proposal, and should instead be realised within a framework based on Contracts. This paper is proposing that framework.

The paper \cite{P3599R0} proposed a subset of this paper for C++26, covering just the runtime checks also proposed by \cite{P3081R2} for C++26, as a counter-proposal to that paper. Since Profiles did not make it into C++26, \cite{P3229R0} is obsolete; our approach for these (and all other) runtime checks is now being proposed in this paper and is targeting the core language UB white paper.

The paper \cite{P3229R0} proposed another subset of this paper for C++26, covering just the changes to \emph{erroneous behaviour} that are needed to make it compatible with our approach. That paper was reviewed, but not accepted by EWG at the Hagenberg meeting; a revised approach to erroneous behaviour is now being proposed in Section~\ref{eb} of this paper.

Finally, this paper also relates to --- and proposes a solution for --- another much-discussed issue: the fact that \emph{explicit} contract assertions in C++26, as specified in \cite{P2900R14}, can themselves have undefined behaviour when checked, because explicit contract assertion predicates are boolean expressions and thus follow the usual rules for evaluating expressions in C++. This property has been repeatedly raised as a concern (see \cite{P2680R1}, \cite{P3173R0}, \cite{P3285R0}, and \cite{P3362R0}). However, the approach suggested in those papers --- constrain contract assertion predicates to expressions that can be statically proven to have no undefined behaviour --- does not seem to be specifiable, implementable, or usable in practice (see  \cite{P3376R0}, \cite{P3386R0}, and \cite{P3499R1}). What \emph{does} work to solve this problem is to specify a framework for runtime mitigation of undefined behaviour across the \emph{entire} core language specification. It will then automatically also apply to contract assertion predicates. This paper is that framework.

\section{Categories of undefined behaviour}
\label{categoriesofub}

% TIMUR: START WRITING THIS SECTION FIRST ON TUE 8 APR 2025 - EVERYTHING ELSE WILL FALL INTO PLACE

TODO

\subsection{Language undefined behaviour vs. library undefined behaviour}

TODO


- Hardening
  - We don't talk about library undefined behaviour, only about language ub
  - refer to impl choice and library hardening for contract assertions in library

\subsection{Impact and potential damage}

TODO

\subsection{Static vs. dynamic mitigation}

TODO

\subsection{Easy to check vs. difficult to check}
\label{howeasytocheck}

TODO

\subsection{Cheap to check vs. expensive to check}

TODO

\subsection{Does a ``safe'' fallback behaviour exist?}

TODO

\section{Erroneous behaviour}
\label{eb}

Repeat points from P3229

\section{Proposed design}

TODO

\subsection{Proposed implicit contract assertions}
\label{proposedlist}

\subsection{Semantic selection}
\label{labels}

TODO


\subsection{Library API}
\label{libraryapi}

TODO

\section{Proposed wording}

TODO


\section*{Document history}
\begin{itemize}
\item \textbf{R0}, 2023-03-08: Initial version.
\item \textbf{R1}, 2024-10-16: Complete rewrite after the WG21 meeting in St. Louis.
\item \textbf{R2}, 2025-04-16: Complete rewrite after the WG21 meeting in Hagenberg and the addition of \cite{P2900R14} into the C++26 working paper.
\end{itemize}

\section*{Acknowledgements}

We wish to thank Herb Sutter, Oliver Rosten, Andrzej Krzemie\' nski, and Roger Orr for their helpful feedback on this paper.

%%%%%%%%%%%%%%%%%%%%%%%%%%%%%%%%%%%%%%%%%%%%%

% Remove ToC entry for bibliography
\renewcommand{\addcontentsline}[3]{}% Make \addcontentsline a no-op to disable auto ToC entry

%\renewcommand{\bibname}{References}  % custom name for bibliography
\bibliographystyle{abstract}
\bibliography{ref}

%%%%%%%%%%%%%%%%%%%%%%%%%%%%%%%%%%%%%%%%%%%%%

\end{document}

